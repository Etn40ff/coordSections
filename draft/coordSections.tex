\documentclass{amsart}

\usepackage[T1]{fontenc}
\usepackage{enumerate, amsmath, amsfonts, amssymb, amsthm, mathrsfs, wasysym, graphics, graphicx, xcolor, url, hyperref, hypcap,  shuffle, xargs, multicol, overpic, pdflscape, multirow, hvfloat, minibox, accents, array, xifthen, a4wide, ae, aecompl}
\usepackage{marginnote}
\hypersetup{colorlinks=true, citecolor=darkblue, linkcolor=darkblue}
\usepackage[all]{xy}
\usepackage[bottom]{footmisc}
\usepackage{tikz}
\usepackage{tkz-graph}
\usetikzlibrary{trees, decorations, decorations.markings, shapes, arrows, matrix, calc, fit, intersections, patterns}
\graphicspath{{figures/}}
\usepackage{caption}
\captionsetup{width=\textwidth}

%%%%%%%%%%%%%%%%%%%%%%%%%%%%%%%%%%%%%%

\title[Geometric realizations of the accordion complex of a dissection]{Geometric realizations of the \\ accordion complex of a dissection}

\thanks{Partially supported by the French ANR grant SC3A~(15\,CE40\,0004\,01).}

\author{Thibault Manneville}
\address[Thibault Manneville]{LIX, \'Ecole Polytechnique}
\email{thibault.manneville@lix.polytechnique.fr}
\urladdr{\url{http://www.lix.polytechnique.fr/~manneville/}}

\author{Vincent Pilaud}
\address[Vincent Pilaud]{CNRS \& LIX, \'Ecole Polytechnique, Palaiseau}
\email{vincent.pilaud@lix.polytechnique.fr}
\urladdr{\url{http://www.lix.polytechnique.fr/~pilaud/}}

%%%%%%%%%%%%%%%%%%%%%%%%%%%%%%%%%%%%%%

% theorems
\newtheorem{theorem}{Theorem}%[section]
\newtheorem{corollary}[theorem]{Corollary}
\newtheorem{proposition}[theorem]{Proposition}
\newtheorem{lemma}[theorem]{Lemma}
\newtheorem{conjecture}[theorem]{Conjecture}

\theoremstyle{definition}
\newtheorem{definition}[theorem]{Definition}
\newtheorem{example}[theorem]{Example}
\newtheorem{remark}[theorem]{Remark}
\newtheorem{question}[theorem]{Question}

% math special letters
\newcommand{\R}{\mathbb{R}} % reals
\newcommand{\N}{\mathbb{N}} % naturals
\newcommand{\Z}{\mathbb{Z}} % integers
\newcommand{\C}{\mathbb{C}} % complex
\newcommand{\I}{\mathbb{I}} % set of integers
\newcommand{\HH}{\mathbb{H}} % hyperplane
\newcommand{\fA}{\mathfrak{A}} % alternating group
\newcommand{\fS}{\mathfrak{S}} % symmetric group
\newcommand{\cA}{\mathcal{A}} % algebra
\newcommand{\cC}{\mathcal{C}} % collection
\newcommand{\rmX}{{\rm X}} % X
\renewcommand{\b}[1]{\mathbf{#1}} % bold letters
\newcommand{\h}{\widehat} % hat letters

% math commands
\newcommand{\set}[2]{\left\{ #1 \;\middle|\; #2 \right\}} % set notation
\newcommand{\setangle}[2]{\left\langle #1 \;\middle|\; #2 \right\rangle} % set notation
\newcommand{\bigset}[2]{\big\{ #1 \;\big|\; #2 \big\}} % big set notation
\newcommand{\biggset}[2]{\bigg\{ #1 \;\bigg|\; #2 \bigg\}} % bigg set notation
\newcommand{\ssm}{\smallsetminus} % small set minus
\newcommand{\dotprod}[2]{\left\langle \, #1 \; \middle| \; #2 \, \right\rangle} % dot product
\newcommand{\symdif}{\,\triangle\,} % symmetric difference
\newcommand{\one}{{1\!\!1}} % the all one vector
\newcommand{\eqdef}{\mbox{\,\raisebox{0.2ex}{\scriptsize\ensuremath{\mathrm:}}\ensuremath{=}\,}} % :=
\newcommand{\defeq}{\mbox{~\ensuremath{=}\raisebox{0.2ex}{\scriptsize\ensuremath{\mathrm:}} }} % =:
\newcommand{\polar}{^\diamond} % polar
\newcommand{\simplex}{\triangle} % simplex
\renewcommand{\implies}{\Rightarrow} % imply sign
\newcommand{\transpose}[1]{{#1}^t} % transpose matrix

% polytopes
\newcommand{\Asso}{\mathsf{Asso}} % associahedron
\newcommand{\Acco}{\mathsf{Acco}} % accordiohedron
\newcommand{\Perm}{\mathsf{Perm}} % permutahedron
\newcommand{\Para}{\mathsf{Para}} % parallelepiped
\newcommand{\Zono}{\mathsf{Zono}} % zonotope
\DeclareMathOperator{\face}{\mathbf{F}} % face of the permutahedron
\newcommand{\Fan}{\mathcal{F}} % fan
\newcommand{\Cone}{\mathrm{C}} % cone

% operators
\DeclareMathOperator{\conv}{conv} % convex hull
\DeclareMathOperator{\vect}{vect} % linear span
\DeclareMathOperator{\cone}{cone} % cone hull
\DeclareMathOperator{\inv}{inv} % inversions
\DeclareMathOperator{\des}{des} % descents

% others
\newcommand{\fref}[1]{Figure~\ref{#1}} % reference figures
\newcommand{\ie}{\textit{i.e.}~} % id est
\newcommand{\eg}{\textit{e.g.}~} % exempli gratia
\newcommand{\Eg}{\textit{E.g.}~} % exempli gratia
\newcommand{\viceversa}{\textit{vice versa}} % vice versa
\newcommand{\versus}{\textit{vs.}~} % versus
\newcommand{\aka}{\textit{a.k.a.}~} % also known as
\newcommand{\perse}{\textit{per se}} % per se
\newcommand{\ordinal}{\textsuperscript{th}} % th for ordinals
\newcommand{\ordinalst}{\textsuperscript{st}} % st for ordinals
\definecolor{darkblue}{rgb}{0,0,0.7} % darkblue color
\newcommand{\darkblue}{\color{darkblue}} % darkblue command
\newcommand{\defn}[1]{\textsl{\darkblue #1}} % emphasis of a definition
\newcommand{\para}[1]{\medskip\noindent\textbf{#1.}} % paragraph
\renewcommand{\topfraction}{1} % possibility to have one page of pictures
\renewcommand{\bottomfraction}{1} % possibility to have one page of pictures
\renewcommand\labelitemi{$\diamond$} % redefine itemize default symbol

% marginal comments
\usepackage{todonotes}
\newcommand{\vincent}[1]{\todo[color=blue!30]{#1 \\ \hfill --- V.}}
\newcommand{\pierreguy}[1]{\todo[color=red!30]{#1 \\ \hfill --- PG.}}
\newcommand{\salvatore}[1]{\todo[color=green!30]{#1 \\ \hfill --- S.}}

% specific to accordion complex
% accordion complex, lattice, flip graph
\newcommandx{\accordionComplex}[1][1=\dissection_\circ]{\mathcal{AC}(#1)} % accordion complex of a dissection
% dissections
\newcommand{\polygon}{\mathrm{P}} % polygon
\newcommand{\triangulation}{\mathrm{T}} % triangulation
\newcommand{\quadrangulation}{\mathrm{Q}} % quadrilateral
\newcommand{\dissection}{\mathrm{D}} % dissection
\newcommand{\cell}{\mathrm{C}} % cell in dissection
\newcommand{\quadrilateral}{\mathrm{Q}} % quadrilateral in a dissection
\newcommand{\accordion}{\mathrm{A}} % accordion in a dissection
\newcommand{\zigzag}{\mathrm{Z}} % zigzag
\newcommand{\snake}{\reflectbox{$\mathrm{Z}$}} % snake
\newcommand{\tree}{\mathrm{T}} % tree
% signatures
\newcommand{\sign}[3]{\varepsilon \big( {#1} \in {#2}\;|\;{#3} \big)} % sign
\newcommand{\SSS}{\reflectbox{$\mathsf{Z}$}} % S
\newcommand{\sss}{\reflectbox{\tiny$\mathsf{Z}$}} % S
\newcommand{\ZZZ}{\mathsf{Z}} % Z
\newcommand{\zzz}{\text{\tiny$\mathsf{Z}$}} % Z
\newcommand{\VVV}{{\mathsf{{V \hspace{-.1686cm} I\,}}}} % V
\newcommand{\signature}{\varepsilon} % signature
% g-vectors
\newcommand{\gvector}[2]{\mathbf{g}(#1 \,|\, #2)} % g-vector of the cluster variable #2 with respect to the initial cluster #1
\newcommand{\biggvector}[2]{\mathbf{g} \big( #1 \,|\, #2 \big)} % g-vector of the cluster variable #2 with respect to the initial cluster #1
\newcommand{\gvectors}[2]{\mathbf{g}(#1 \,|\, #2)} % g-vectors of the cluster #2 with respect to the initial cluster #1
\newcommand{\biggvectors}[2]{\mathbf{g} \big( #1 \,|\, #2 \big)} % g-vectors of the cluster  #2 with respect to the initial cluster #1
\newcommand{\gvectorFan}{\mathcal{F}^\mathbf{g}} % g-vector fan
\newcommand{\bg}[1]{\mathbf{g}(#1)}
% d-vectors
\newcommand{\comp}[2]{(#1 \,|\, #2)} % compatibility
\newcommand{\bigcomp}[2]{(#1 \,|\, #2)} % compatibility
\newcommand{\dvector}[2]{\mathbf{d}(#1 \,|\, #2)} % d-vector of the cluster variable #2 with respect to the initial cluster #1
\newcommand{\bigdvector}[2]{\mathbf{d} \big( #1  \,|\, #2 \big)} % d-vector of the cluster variable #2 with respect to the initial cluster #1
\newcommand{\dvectors}[2]{\mathbf{d}(#1 \,|\, #2)} % d-vectors of the cluster #2 with respect to the initial cluster #1
\newcommand{\bigdvectors}[2]{\mathbf{d} \big( #1  \,|\, #2 \big)} % d-vectors of the cluster #2 with respect to the initial cluster #1
\newcommand{\dvectorFan}{\mathcal{F}^\mathbf{d}} % d-vector fan
% c-vectors
\newcommand{\cvector}[3]{\mathbf{c}(#1  \,|\, #3 \in #2)} % c-vector of the cluster variable #3 in the cluster #2 with respect to the initial cluster #1
\newcommand{\bigcvector}[3]{\mathbf{c} \big( #1  \,|\, #3 \in #2 \big)} % c-vector of the cluster variable #3 in the cluster #2 with respect to the initial cluster #1
\newcommand{\cvectors}[2]{\mathbf{c}(#1  \,|\, #2)} % c-vectors of the cluster #2 with respect to the initial cluster #1
\newcommand{\bigcvectors}[2]{\mathbf{c} \big( #1  \,|\, #2 \big)} % c-vectors of the cluster #2 with respect to the initial cluster #1
\newcommand{\allcvectors}[1]{\mathbf{C}(#1)} % all c-vectors with respect to the initial cluster #1
\newcommand{\cvectorFan}{\mathcal{F}^\mathbf{c}} % fan of hyperplanes orthogonal to all c-vectors
% rhs
\newcommand{\rhs}[2]{\omega(#1 \,|\, #2)} % right hand side of zonotope
\newcommand{\bigrhs}[2]{\omega \big( #1  \,|\, #2 \big)} % right hand side of zonotope
\newcommand{\rhsZZZ}[2]{\omega_{\zzz}(#1 \,|\, #2)} % right hand side of zonotope
\newcommand{\bigrhsZZZ}[2]{\omega_{\zzz} \big( #1  \,|\, #2 \big)} % right hand side of zonotope
\newcommand{\rhsSSS}[2]{\omega_{\sss}(#1 \,|\, #2)} % right hand side of zonotope
\newcommand{\bigrhsSSS}[2]{\omega_{\sss} \big( #1  \,|\, #2 \big)} % right hand side of zonotope
\newcommand{\rhstilde}[1]{\widetilde\omega(#1)} % right hand side of projected zonotope
% points, hyperplanes, half-spaces
\newcommand{\point}[2]{\mathbf{p}(#1  \,|\, #2)} % vertex of the #1-associahedron corresponding to the cluster #2
\newcommand{\bigpoint}[2]{\mathbf{p} \big( #1  \,|\, #2 \big)} % vertex of the #1-associahedron corresponding to the cluster #2
\newcommand{\pointtilde}[2]{\widetilde{\mathbf{p}}(#1  \,|\, #2)} % vertex of the #1-associahedron corresponding to the cluster #2
\newcommand{\ray}{\mathbf{r}} % ray
\newcommand{\rays}{\mathbf{R}} % rays
\newcommand{\hs}{\mathbf{H}^{\le}} % half space
\newcommand{\HS}[2]{\mathbf{H}^{\le}(#1  \,|\, #2)} % half space
\newcommand{\bigHS}[2]{\mathbf{H}^{\le} \big( #1  \,|\, #2 \big)} % half space
\newcommand{\HStilde}[2]{\widetilde{\mathbf{H}}^{\le}(#1  \,|\, #2)} % half space
\newcommand{\hyp}{\mathbf{H}^{=}} % hyperplane
\newcommand{\Hyp}[2]{\mathbf{H}^{=}(#1 \,|\, #2 )} % hyperplane
\newcommand{\bigHyp}[2]{\mathbf{H}^{=} \big( #1  \,|\, #2 \big)} % hyperplane
\newcommand{\fix}[1]{\mathrm{Fix}(#1)} % fix space
% other
\newcommand{\quiver}{\mathrm{Q}} % quiver
\newcommand{\CoxeterGroup}{\mathrm{W}} % Coxeter group
\newcommand{\mi}{-} % bottom dissection
\newcommand{\ma}{+} % top dissection
\newcommand{\ini}{\mathrm{ini}} % initial dissection
\newcommand{\ex}{\mathrm{ex}} % example dissection
\newcommand{\projection}{\pi} % projection
\newcommand{\restrictedComplex}[3]{\Delta^{\b{#1}}(#2,#3)} % restricted complex in cluster algebras
\renewcommand{\restriction}[2]{\left.\kern-\nulldelimiterspace #1 \vphantom{\big|} \right|_{#2}}

% silting complex
\newcommandx{\siltingComplex}[1][1=\quiver]{\mathcal{SC}(#1)}

%%%%%%%%%%%%%%%%%%%%%%%%%%%%%%%%%%%%%%

\begin{document}

\begin{abstract}
Consider $2n$ points on the unit circle and a reference dissection~$\dissection_\circ$ of the convex hull of the odd points. The accordion complex of~$\dissection_\circ$ is the simplicial complex of non-crossing subsets of the diagonals with even endpoints that cross an accordion of the dissection~$\dissection_\circ$. In particular, this complex is an associahedron when~$\dissection_\circ$ is a triangulation and a Stokes complex when~$\dissection_\circ$ is a quadrangulation. In this paper, we provide geometric realizations (by polytopes and fans) of the accordion complex of any reference dissection~$\dissection_\circ$, generalizing known constructions arising from cluster algebras.

\medskip
\noindent
\textsc{keywords.} Permutahedra $\cdot$ Zonotopes $\cdot$ Associahedra $\cdot$ $\b{g}$-, $\b{c}$- and $\b{d}$-vectors.
\end{abstract}

\vspace*{-1cm}


%%%%%%%%%%%%%%%%%%%%%%%%%%%%%%%%%%%%%%

\section{Motivating example: accordion complexes of dissections}

Let~$\polygon$ be a convex polygon.
We call \defn{diagonals} of~$\polygon$ the segments connecting two non-consecutive vertices of~$\polygon$.
A \defn{dissection} of~$\polygon$ is a set~$\dissection$ of non-crossing diagonals.
It dissects the polygon into \defn{cells}.
%We denote by~$Q(\dissection)$ the quiver with relations having a vertex for each diagonal of~$\dissection$, an arrow connecting any two counterclockwise consecutive edges of a cell of~$\dissection$, and a relation between any two arrows connecting three counterclockwise consecutive edges of a cell~of~$\dissection$.
We denote by~$\quiver(\dissection)$ the quiver with relations whose vertices are the diagonals of~$\dissection$, whose arrows connect any two counterclockwise consecutive edges of a cell of~$\dissection$, and whose relations are given by triples of counterclockwise consecutive edges of a cell of~$\dissection$.
See \fref{} for an example.

We now consider $2m$ points on the unit circle alternatly colored black and white, and let~$\polygon_\circ$ (resp.~$\polygon_\bullet$) denote the convex hull of the white (resp.~black) points.
We fix an arbitrary reference dissection~$\dissection_\circ$ of~$\polygon_\circ$.
A solid diagonal~$\delta_\bullet$ of~$\polygon_\bullet$ is a \defn{$\dissection_\circ$-accordion diagonal} if it does not enter and exit any cell of~$\dissection_\circ$ crossing two non-consecutive of its edges.
In other words the diagonals of~$\dissection_\circ$ crossed by~$\delta_\bullet$ form an accordion.
A \defn{$\dissection_\circ$-accordion dissection} is a set of non-crossing $\dissection_\circ$-accordion diagonals. We call \defn{$\dissection_\circ$-accordion complex} the simplicial complex~$\accordionComplex$ of $\dissection_\circ$-accordion dissections.

%Consider a diagonal~$\delta_\circ$ of~$\dissection_\circ$ and~$\delta_\bullet$ be a $\dissection_\circ$-accordion diagonal intersecting~$\delta_\circ$.
%Let~$\mu_\circ$ and~$\nu_\circ$ be the two other edges crossed by~$\delta_\bullet$ in the two cells of~$\dissection_\circ$ containing~$\delta_\circ$.
%Since~$\delta_\bullet$ be a $\dissection_\circ$-accordion diagonal, $\mu_\circ$ and~$\nu_\circ$ are incident to~$\delta_\circ$.
%We define~$\sign{\delta_\circ}{\dissection_\circ}{\delta_\bullet}$ to be $1$, $-1$, or~$0$ depending on whether~$\mu_\circ \delta_\circ \nu_\circ$ forms a~$\ZZZ$, a~$\SSS$, or a~$\VVV$. 

%Consider a diagonal~$\delta_\circ$ of~$\dissection_\circ$.
%For a $\dissection_\circ$-accordion diagonal~$\delta_\bullet$ intersecting~$\delta_\circ$, the two other edges crossed by~$\delta_\bullet$ in the two cells of~$\dissection_\circ$ containing~$\delta_\circ$ are incident to~$\delta_\circ$.
%We define~$\sign{\delta_\circ}{\dissection_\circ}{\delta_\bullet}$ to be $1$, $-1$, or~$0$ depending on whether these two edges together with~$\delta_\circ$ form a~$\ZZZ$, a~$\SSS$, or a~$\VVV$. 

For a diagonal~$\delta_\circ$ of~$\dissection_\circ$ and a $\dissection_\circ$-accordion diagonal~$\delta_\bullet$ intersecting~$\delta_\circ$, we consider the three edges (including~$\delta_\circ$) crossed by~$\delta_\bullet$ in the two cells of~$\dissection_\circ$ containing~$\delta_\circ$. We define~$\sign{\delta_\circ}{\dissection_\circ}{\delta_\bullet}$ to be $1$, $-1$, or~$0$ depending on whether these three edges form a~$\ZZZ$, a~$\SSS$, or a~$\VVV$.
The \defn{$\b{g}$-vector} of~$\delta_\bullet$ with respect to~$\dissection_\circ$ is the vector~$\gvector{\dissection_\circ}{\delta_\bullet} \in \R^{\dissection_\circ}$ whose $\delta_\circ$-coordinate is~$\sign{\delta_\circ}{\dissection_\circ}{\delta_\bullet}$.
%The collection of cones
%\[
%\gvectorFan(\dissection_\circ) \eqdef \bigset{\R_{\ge0} \biggvectors{\dissection_\circ}{\dissection_\bullet}}{\dissection_\bullet \text{ any $\dissection_\circ$-accordion dissection}}
%\]
%forms a complete simplicial fan, that we call t.
For example, ...
\vincent{Do example}

\begin{example}
\label{exm:associahedron}
When the reference dissection~$\dissection_\circ$ is a triangulation of~$\polygon_\circ$, any diagonal of~$\polygon_\bullet$ is a $\dissection_\circ$-accordion diagonal.
The $\dissection_\circ$-accordion complex is thus an $n$-dimensional associahedron (of type~$A$), where~$n = m-3$.
In this case, it is known that the $\dissection_\circ$-accordion complex is isomorphic to the $2$-term silting complex of the quiver~$\quiver(\dissection_\circ)$ of the triangulation~$\dissection_\circ$ (see Section~\ref{sec:PGP}~for~definitions).
\end{example}

The initial motivation of this paper was to prove the following extension of Example~\ref{exm:associahedron}.

\begin{theorem}
\label{thm:bijectionAccordionComplexSiltingComplex}
For any reference dissection~$\dissection_\circ$, the $\dissection_\circ$-accordion complex is isomorphic to the $2$-term silting complex of the quiver~$\quiver(\dissection_\circ)$.
\end{theorem}

One possible approach to Theorem~\ref{thm:bijectionAccordionComplexSiltingComplex} would be to provide an explicit bijective map between $\dissection_\circ$-accordion diagonals and $2$-term projective complexes for~$\quiver(\dissection_\circ)$.
Such a map is easy to guess using $\b{g}$-vectors, but the proof that this map is actually a bijection and that it preserves compatibility is intricated.
This approach was developed in the more general context of non-kissing complexes of gentle quivers in~\cite{PaluPilaudPlamondon}.
In this paper, we use an alternative simpler strategy to obtain Theorem~\ref{thm:bijectionAccordionComplexSiltingComplex}, understanding accordion complexes as certain subcomplexes of the associahedron.

For that, consider two nested dissections~$\dissection_\circ \subset \dissection_\circ'$.
Observe that any~$\dissection_\circ$-accordion diagonal is a $\dissection_\circ'$-accordion diagonal.
Conversely a $\dissection_\circ'$-accordion diagonal~$\delta_\bullet$ is a $\dissection_\circ$-accordion diagonal if and only if it does not cross any diagonal of~$\dissection_\circ' \ssm \dissection_\circ$ as a $\ZZZ$ or a~$\SSS$, that is if and only if its $\b{g}$-vector~$\gvector{\dissection_\circ'}{\delta_\bullet}$ belongs to the subspace spanned by elements in~$\dissection_\circ$.
This observation shows the following statement.

\begin{theorem}[\cite{MannevillePilaud-accordion}]
\label{thm:contractDiagonals}
%Consider two dissections~$\dissection_\circ \subset \dissection_\circ'$. A $\dissection_\circ'$-accordion diagonal~$\delta_\bullet$ is also a $\dissection_\circ$-accordion diagonal if and only if ${\sign{\delta'_\circ}{\dissection_\circ}{\delta_\bullet}}$ vanishes for all~$\delta'_\circ \in \dissection_\circ' \ssm \dissection_\circ$.
For any two nested dissections~$\dissection_\circ \subset \dissection_\circ'$, the accordion complex~$\accordionComplex$ is isomorphic to the subcomplex of~$\accordionComplex[\dissection_\circ']$ induced by $\dissection_\circ'$-accordion diagonals~$\delta_\bullet$ whose $\b{g}$-vector~$\gvector{\dissection_\circ'}{\delta_\bullet}$ lie in the coordinate subspace spanned by elements in~$\dissection_\circ$.
\end{theorem}

Consider now any quiver with relations~$\quiver$ and any subset~$J$ of vertices of~$\quiver$.
We call \defn{shortcut quiver} the quiver with relations~$\quiver/J$ whose vertices are the vertices of~$\quiver$ not in~$J$, whose arrows are the paths in~$\quiver$ with internal vertices in~$J$, and whose relations are inherited from those of~$\quiver$.
For example, quivers of subdissections are shortcut quivers: if~$\dissection_\circ \subset \dissection_\circ'$, then~${\quiver(\dissection_\circ) = \quiver(\dissection_\circ')/(\dissection_\circ' \ssm \dissection_\circ)}$.
This paper proves the following statement.

\begin{theorem}
\label{thm:contractVertices}
For any quiver with relations~$\quiver$ and any subset~$J$ of vertices of~$\quiver$, the $2$-term silting complex~$\siltingComplex[\quiver/J]$ for the shortcut quiver~$\quiver/J$ is isomorphic to the subcomplex of the $2$-term silting complex~$\siltingComplex$ induced by $2$-term projective complexes whose $\b{g}$-vector lie in the coordinate subspace spanned by vertices not in~$J$.
\end{theorem}

Combining Theorems~\ref{thm:contractDiagonals} and~\ref{thm:contractVertices} together with Example~\ref{exm:associahedron} proves Theorem~\ref{thm:bijectionAccordionComplexSiltingComplex}.

\vincent{Fans}


%%%%%%%%%%%%%%%%%%%%%%%%%%%%%%%%%%%%%%

\section*{Acknoledgements}

We are grateful to F.~Chapoton for various conversations on quadrangulations and Stokes posets.

%%%%%%%%%%%%%%%%%%%%%%%%%%%%%%%%%%%%%%%

\bibliographystyle{alpha}
\bibliography{coordSections}
\label{sec:biblio}

\end{document}
